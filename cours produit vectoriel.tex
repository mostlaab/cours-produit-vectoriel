\documentclass{beamer}
\makeatletter \newcommand{\leftm}{\@totalleftmargin}\makeatother
\newcommand{\RTListe}{\raggedleft\rightskip\leftm}

\usepackage{hyperref}

\usetheme{Frankfurt}%Frankfurt,CambridgeUS,Madrid
\setbeamertemplate{footline}[frame number]
\usefonttheme[]{serif}
%\useoutertheme{miniframes}
\usecolortheme{crane}
\usecolortheme{seahorse}   %seahorse, wolverine,beaver, default, albatross, beaver, crane, dolphin, lily, orchid, rose, whale
\setbeamercovered{transparent} 

\useinnertheme[shadow=true]{rounded}
\usepackage{etoolbox}
%\setbeamercolor{block title}{use=structure,fg=structure.fg,bg=structure.fg!20!bg}
%\setbeamercolor{block body}{parent=normal text,use=block title,bg=block title.bg!50!bg}
%\usecolortheme{wolverine}%,wolverine,albatross,whale,seahorse,rose

\usepackage{tikz,pifont}
\usetikzlibrary{shapes.gates.logic.US,trees,positioning,arrows}
\definecolor{BrickRed}{RGB}{132,31,39}
\usetikzlibrary{decorations.pathreplacing}
\usetikzlibrary{shapes}
\usepackage{tcolorbox}
\usepackage{multicol,color}
\usepackage{varwidth}
\numberwithin{equation}{section}

% for RTL liste
\newcommand{\rr}{\mathbb{R}}\newcommand{\N}{\mathbb{N}}
\newcommand{\Q}{\mathbb{Q}}\newcommand{\Z}{\mathbb{Z}}
\newcommand{\cc}{\mathbb{C}}



% RTL frame title
\setbeamertemplate{frametitle}
{\vspace*{-1mm}
	\nointerlineskip
	\begin{beamercolorbox}[sep=0.4cm,ht=2.0em,wd=\paperwidth]{frametitle}
		\vbox{}\vskip-2ex%
		\strut\hskip1ex\insertframetitle\strut
		\vskip-0.8ex%
	\end{beamercolorbox}
}

% align subsection in toc
\makeatletter
\setbeamertemplate{subsection in toc}
{\leavevmode\rightskip=5ex%
	\llap{\raise0.1ex\beamer@usesphere{subsection number projected}{bigsphere}\kern1ex}%
	\inserttocsubsection\par%
}
\makeatother

% RTL triangle for itemize
\setbeamertemplate{itemize item}{\scriptsize\raise1.25pt\hbox{\donotcoloroutermaths$\blacktriangleleft$}} 

%\setbeamertemplate{itemize item}{\rule{4pt}{4pt}}

\defbeamertemplate{enumerate item}{square2}
{\LR{
		%
		\hbox{%
			\usebeamerfont*{item projected}%
			\usebeamercolor[bg]{item projected}%
			\vrule width2.25ex height1.85ex depth.4ex%
			\hskip-2.25ex%	
			\hbox to2.25ex{%
				\hfil%
				{\color{fg}\insertenumlabel}%
				\hfil}%
		}%
}}

\setbeamertemplate{enumerate item}[square2]

\setbeamertemplate{navigation symbols}{}
\logo{مصطفى لعببدي} %logo texte

\newcommand{\V}[1]{\overrightarrow{#1}}
%==================================
\tcbuselibrary{skins,listings,breakable,xparse}
\tcbset{enhanced,boxrule=1pt,sharp corners,rounded corners=all,bicolor,colback=green!10!white,colbacklower=white,colframe=BrickRed}

\usepackage{newverbs}
\newverbcommand{\cverb}{\color{blue}}{}
%\AtBeginSection[]
%{
%	\begin{frame}{}
%	\tableofcontents[currentsection] %current
%\end{frame}
%}


%\AtBeginSubsection[]
%{
%	\begin{frame}{}
%	\tableofcontents[currentsubsection,hideothers] %current
%\end{frame}
%}
%\setbeameroverlaytransparent
%=================================================================
\author{ذ:	مصطفى لعبيدي }
\title{ \huge	الجداء المتجهي }
\institute[lycee Tarik]{\textarabic{ثانوية طارق بن زيادالتاهيلية \\
بازرو}}
%=================================================================

\newcounter{exocount}
\setcounter{exocount}{0}
\newenvironment{exo}{\stepcounter{exocount}\begin{block}{تمرين ~\theexocount}}%
	{\end{block}}

%===================================================================
\usepackage{polyglossia}
\setdefaultlanguage[locale=algeria]{arabic}
\newfontfamily\arabicfont[Script=Arabic,Scale=1]{Amiri}
\setotherlanguage{french}
\let\arabicfonttt\ttfamily
%===================================================================
%\lstdefinestyle{json}{
%	language=[LaTeX]TeX,
%	,escapeinside=``,
%	keywordstyle=\color{blue},%
%	upquote=true,
%	columns=flexible,
%	texcsstyle=*\color{blue},
%	basicstyle=\scriptsize,
%	stringstyle=\color{blue},
%	commentstyle=\color{red},	moretexcs={xymatrix,setbeamertemplate,usetheme,usefonttheme,,rotatebox,diagbox,newcolumntype,arraybackslash,underbracket,mathclap,closedcycle,setbeamercolor,url,rowcolors,appendix,filldraw,foreach,shade,path,definecolor,color,underset,textcolor,tikzstyle,tikzset,part,RL,LR,setdefaultlanguage,setotherlanguage,newfontfamily,arabicfont,text,iint,iiint,dfrac,tfrac,multirow,href,rhead,cfoot,thesection,lhead,fancyplain,leftmark,chapter,tableofcontents,savebox,maketitle,mathbb,includegraphics, systeme,setlength,contentsname,sysequivsign,sysaddeqsign,labelenumii,labelenumi,alph,textLR,
%		arabicfontsf,arabicfonttt,draw,node, addcontentsline,addplot,numberwithin,bibname,sysautonum,syscodeextracol,sysextracolsign,syslineskipcoeff,efootnote,subsection,subsubsection,abstractname,textsubscript,mathcal}}

%%%%%%%%%%%%%%%%%%%%%%%%%%%%%%%%%%%%%%%%%%%%%%%%%%%%%%%%%%%%%%%%%%%%%%%%%%%%%%%%%%%%%%%%%%%%%%%%%%%%%%%%%%%%%%%%%%%%%%%%%%%%%%%%%%%%%%%%%%%%%%%%%%%%%%%%%%%%%%%%%%%%

%\usepackage{tikz}
\usetikzlibrary{fit,calc,positioning}
\tikzset{%
	highlight/.style={rectangle,rounded corners,fill=red!15,draw,fill opacity=0.5,thick,inner sep=0pt}
}
\newcommand{\tikzmark}[2]{\tikz[overlay,remember picture,baseline=(#1.base)] \node (#1) {#2};}
%
\newcommand{\Highlight}[1][submatrix]{%
	\tikz[overlay,remember picture]{
		\node[highlight,fit=(left.north west) (right.south east)] (#1) {};}
}
\newcommand{\antiHighlight}[1][submatrix]{%
	\tikz[overlay,remember picture]{
		\node[highlight,fit=(left.north west) (right.south east),dashed] (#1) {};}
}

\newcommand{\tikzcontour}[1]{\tikz[overlay,remember picture] \node[red,draw,dotted,opacity=.7,inner sep=0pt] (#1) {#1};}

%%%%%%%%%%%%%%%%%%%%%%%%%%%%%%%%%%%%%%%%%%%%%%%%%%%%%%%%%%%%%%%%%%%%%%%%%%%%%%%%%%%%%%%%%%%%%%%%%%%%%%%%%%%%%%%%%%%%%%%%%%%%%%%%%%%%%%%%%%%%%%%%%%%%%%%%%%%%%%%%%%%%

%\listfiles %si on veut afficher la liste des packages dans le fichier log
\begin{document}
\begin{frame}[plain]\transsplitverticalout
\begin{tcolorbox}[colback=white,drop large lifted shadow,top=.5cm,bottom=.5cm]
	{\begin{center}
			{\large{بسم الله الرحمان الرحيم}} %\emph\ttfamily
			\maketitle
	\end{center}}
\end{tcolorbox}
\end{frame}

\begin{frame}[plain]{\large \textbf{\huge	تتمة درس الهندسة الفضائية	:}}
\transglitter[duration=0.5]

\begin{tcolorbox}[colback=white,drop large lifted shadow,top=.5cm,bottom=.5cm]
\begin{center}
	\begin{LARGE}
		{\color{blue}\huge
			{  الجداء المتجهي - تطبيقات
		}}
	\end{LARGE}
\end{center}
\end{tcolorbox}



\begin{itemize}%[<+->]
	\RTListe %pour aligner à droite les items
\item[$\textcolor{red}{\surd}$] 
تعريف الجداء المتجهي 
\item 
انعدام الجداء المتجهي  واستقامية متجهتين
\item [$\textcolor{red}{\surd}$] 
خاصيات
\item [$\textcolor{red}{\surd}$] 
المعلم المباشر

\item [$\textcolor{red}{\surd}$] 
احداثيات الجداء المتجهي لمتحهتين
\item [$\textcolor{red}{\surd}$] 
تطبيقات  الجداء المتجهي :

\begin{itemize}%[<+->]
	\RTListe
	\item 
	معادلة مستوى معرف بثلاث نقط
	\item 
	مسافة نقطة عن مستقيم
	\item 
	مساحة مثلث
		\item 
	موجهة تقاطع مستويين
\end{itemize}
\end{itemize}
\end{frame}
%page de garde et sommaire

\section{ تعريف }
\begin{frame}{توجيه الفضاء-  توجيه المستوى }
%\title{	الجداء المتجهي}
\begin{minipage} {.75\textwidth}
	

	\begin{small}
		\begin{block}{}
			
		

	ليكن 
	 $\mathcal{R}=(O,\vec{i},\vec{j},\vec{k})$
	 معلما للفضاء أي النقط 
	O,I,J,K 
	 بحيث : \\
	$\vec{i}=\V{OI},\vec{j}=\V{OJ}, \vec{k}=\V{OK}$ 
	نقط غير مستوائية 
	
	رجل امبير للمعلم 
	 $\mathcal{R}$ 
	 هو رجل خيالي راسه في K قدماه في O  وينظر الى I 
	 
	إذاكان J   على يسار هدا الرجل فإن المعلم  يكون مباشراً او موجباً
	
	إداكان المعلم  مباشر فان الاساس
	 	 $(\vec{i},\vec{j},\vec{k})$
	 	 مباشر
	 	 
	الفضاء 
	$(\mathcal{E})$
	 موجه إذاكان منسوباً إلى معلم مباشر
	 
	ملحوظة :
	
	‫يمكن إنشاء  معلم ‬مباشر 
		 $\mathcal{R}=(O,\vec{i},\vec{j},\vec{k})$ 
		  بحيث‬‬ 
$ 	\vec{i},\vec{j} $
	موجهتين  لمستوى 
%	$ (\mathcal{P}) $

	وتوجيه المستوى بتوجيه منظمية عليه 
\end{block}


%\begin{tikzpicture}[remember picture,overlay]
%%	\node at ([xshift=-5cm]current page){ ooo};
%	\begin{scope}[xshift=1cm,yshift=3cm]
%\draw[->,latex] (0,0)node[left]{\tiny O}--node[pos=0.5,left]{$\vec{ k}$} (0,1) node[above]{\tiny K} ;
%\draw[->,latex]		(0,0)-- node[pos=0.5,below]{$\vec{ j}$}(1,0)node[right]{\tiny J} ;
%\draw[->,latex]		(0,0)-- node[pos=0.5,below]{$\vec{ i}$}(-.5,-.5)node[left]{\tiny I};
%
%	\end{scope}
%\end{tikzpicture}
\end{small}
\end{minipage}
\end{frame}
\begin{frame}

\begin{minipage}{.75\textwidth}

	\begin{block}{تعريف  الجداء المتجهي :}
	
	ليكن 
	$ 	\vec{u}=\V{OA},\vec{v}=\V{OB}$
	 متجهتين من الفضاء المتجهي
	  
	الجداء المتجهي للمتجهتين 
		$ 	\vec{u}$ و $\vec{v}$
	بهذا الترتيب هو المتجهة 
	 
	$	\vec{w}= \vec{u} \land \vec{v}$
	
	 المعرفة بمايلي:
	 
	 		$	\vec{w}=\V{OC}$
	 		  
	 			  \begin{enumerate}\RTListe
	 			\item 
	اذاكان 	
	$ 	\vec{u}$ و $\vec{v}$ 
	مستقيميتان فان
		 		$	\vec{w}=\vec{0}$
		 		\item 
	اذاكان 
		$ 	\vec{u}$ و $\vec{v}$
	 غير مستقيميتين فان المتجهة
$\vec{w} $
	  تحقق
	
	  \begin{enumerate}\RTListe
	  	\item 
	$ (OA) 	 \perp	(OC)  $ و	$ (OC) \perp  (OB) $
	\item 
	$  (	\vec{u},\vec{v},\vec{w}) $
	اساس مباشر 
	\item 
$ 	OC= OA \times OB \times | \sin ( \V{OA} ,\V{OB} )| $
	  \end{enumerate}
	  \end{enumerate}

	\end{block}
	
\end{minipage}	
	
%	$ \vec{ i}   \land \vec{j }  = \vec{k } $ \\
%$	\vec{j }  \land  \vec{k } = \vec{i } $\\
%$	\vec{k }  \land  \vec{i }  = \vec{ j} $

%	احسب دائرية جداءات متجهات الاساس
%	التاويل الهندسي لمنظم الجداء المتجهي هو مساحة متوازي اضلاع و منه مساحة مثلث 
\end{frame}
\begin{frame}{خاصيات الجداء المتجهي }
\begin{small}
\begin{minipage}{.75\textwidth}
	
\begin{block}{خاصية}
	\centering
$ \forall \alpha ,\beta \in \rr ,\forall \vec{ u},\vec{ v},\vec{ w} \in \mathcal{V}_3 $ \vspace{.5cm}
\begin{enumerate}[t]\RTListe
	\setlength \itemsep{.5cm}
\item 
$ \vec{u }  \land\vec{v }  = -\vec{v } \land\vec{u }  $
\item 
$ \alpha\beta (\vec{ u} \land\vec{v } )=(\alpha \vec{ u} ) \land(\beta\vec{v })     $  
\item 
$ \vec{u }  \land(\vec{ v}  + \vec{ w}  )=\vec{ u}  \land\vec{v }  + \vec{u}  \land \vec{ w} $
\end{enumerate}
\end{block}
\end{minipage}


ملحوظة : 

اذا كان  
$(\vec{i},\vec{j},\vec{k})$
اساس متعامد ممنظم ومباشر  فإن :

\begin{tabular}{l|l}
%	\hline 
	$ \vec{ i}   \land \vec{i } =\vec{ 0} $	& 	$ \vec{ i}   \land \vec{j }  = \vec{k } $ \\ 
	%		\hline 
	$ \vec{ j}   \land \vec{j } =\vec{ 0}  $	& $	\vec{j }  \land  \vec{k } = \vec{i } $ \\ 
	%		\hline 
	$ \vec{ k}   \land \vec{k } =\vec{ 0}  $	&$	\vec{k }  \land  \vec{i }  = \vec{ j} $  \\ 
%	\hline 
\end{tabular} 

\end{small}
\end{frame}

\begin{frame}[allowframebreaks]{تطبيق :
	الصيغة التحليلية للجداء المتجهي في م.م.م.م }

\begin{block}{}
	

	تحديد احداثيات :
$ 	  \vec{u}\wedge \vec{v}= \begin{pmatrix}u_{1}\\u_{2}\\u_{3}\
\end{pmatrix}\land   \begin{pmatrix}v_{1}\\v_{2}\\v_{3}\\	\end{pmatrix} $
في معلم متعامد ممنظم مباشر  $\mathcal{R}=(O,\vec{i},\vec{j},\vec{k})$

\end{block}
\begin{small}


		\begin{align*}
	\displaystyle \vec{u}\wedge \vec{v}=& (u_{1}\vec{i}+u_{2}\vec{j}+u_{3}\vec{k})
\land  (v_{1}\vec{i}+v_{2}\vec{j}+v_{3}\vec{k})\\
	=&\stackrel{\vec{ 0}}{\overbrace{u_{1}\vec{i}\land v_{1}\vec{i}}} +\stackrel{u_{1}v_{2}\vec{ k}}{\overbrace{u_{1}\vec{i}\land v_{2}\vec{j}}}+\stackrel{-u_{1}v_{3}\vec{j}}{\overbrace{u_{1}\vec{i}\land v_{3}\vec{k}}}+\\
	 &+\stackrel{-u_{2}v_{1}\vec{ k}}{\overbrace{u_{2}\vec{j}\land v_{1}\vec{i}}} +\stackrel{\vec{ 0}}{\overbrace{u_{2}\vec{j}\land v_{2}\vec{j}}}+\stackrel{u_{2}v_{3}\vec{ i}}{\overbrace{u_{2}\vec{j}\land v_{3}\vec{k}}}+\\	
	 &+\stackrel{u_{3}v_{1}\vec{ j}}{\overbrace{u_{3}\vec{k}\land v_{1}\vec{i}}} +\stackrel{-u_{3}v_{2}\vec{ i}}{\overbrace{u_{3}\vec{k}\land v_{2}\vec{j}}}+\stackrel{\vec{ 0}}{\overbrace{u_{3}\vec{k}\land v_{3}\vec{k}}}\\
	=&{(u_{2}v_{3}-u_{3}v_{2})\vec{i}+(\u_{3}v_{1}-u_{1}v_{3})\vec{j}+(u_{1}v_{2}-u_{2}v_{1})\vec{k}}	\\
%	\end{align*}	
%	ومنه نستنتج :
%	
%		\begin{align*}
	=&\left |\begin{array}{cc} u_{2}&v_{2}\\u_{3}&v_{3}\end{array}\right |\vec{i}- \left |\begin{array}{cc}u_{1}&v_{1}\\u_{3}&v_{3}\end{array}\right |\vec{j}+ \left |\begin{array}{cc}  u_{1}&v_{1} \\u_{2}&v_{2}\end{array}\right |\vec{k} \\
	=&{\begin{pmatrix}u_{2}v_{3}-u_{3}v_{2}\\u_{3}v_{1}-u_{1}v_{3}\\u_{1}v_{2}-u_{2}v_{1}\end{pmatrix}}		
	\end{align*}
\end{small}
\end{frame}
%\section{ تعريف }
%% تعريف وامثلة مع الشكل 
\begin{frame}%[plain]%{احدثيات الجداء المتجهي}
\begin{block}{احداثيات الجداء المتجهي في .م.م.م.م}
	\begin{align*}
\displaystyle \vec{u}\wedge \vec{v}=& \begin{pmatrix}u_{1}\\u_{2}\\u_{3}\\
\end{pmatrix}\land   \begin{pmatrix}v_{1}\\v_{2}\\v_{3}\\	\end{pmatrix}\\
=&\left |\begin{array}{cc} u_{2}&v_{2}\\u_{3}&v_{3}\end{array}\right |\vec{i}- \left |\begin{array}{cc}u_{1}&v_{1}\\u_{3}&v_{3}\end{array}\right |\vec{j}+ \left |\begin{array}{cc}  u_{1}&v_{1} \\u_{2}&v_{2}\end{array}\right |\vec{k} \\
=&{\begin{pmatrix}u_{2}v_{3}-u_{3}v_{2}\\u_{3}v_{1}-u_{1}v_{3}\\u_{1}v_{2}-u_{2}v_{1}\end{pmatrix}}		
\end{align*}
%%% le lien suivant ne marche pas dans https://lewebpedagogique.com/most
%	\href{https://upload.wikimedia.org/wikipedia/commons/thumb/b/b2/Technique_de_calcul_du_produit_vectoriel.gif/220px-Technique_de_calcul_du_produit_vectoriel.gif}{ \fbox{انقرهنا }}
\end{block}
\end{frame}
\begin{frame}{مثال :}
\begin{block}{}
	 
	\begin{center}
	 $\vec{u} =\begin{pmatrix}{1}\\{3}\\{5}\\
	\end{pmatrix}\land   \begin{pmatrix}{2}\\{4}\\{6}\\	\end{pmatrix}= \begin{pmatrix}{M_1}\\{M_2}\\{M_3}\\ \end{pmatrix} $
\end{center}
\end{block}
حيث
\LR{
	\[
	M_1 = \left(\begin{array}{*2{c}}
	\tikzcontour{1} &\tikzcontour {2 }\\
	\tikzmark{left} {3} & 4  \\
	5 &  \tikzmark{right}{6}\\
	\end{array}\right) \Highlight[first]
	\quad
	M_2 = \left(\begin{array}{*2{c}}
	\tikzmark{left}   {1} & 2 \\
	\tikzcontour{3} & \tikzcontour{4} \\
	5 &  \tikzmark{right}{6}\\
	\end{array}\right)  \Highlight[second]
	\tikz[overlay,remember picture] {
		\draw[->,thick,red,] ([shift={(-.1cm,.1cm)}]left.south east) -- ([shift={(.1cm,-.2cm)}]right.north west) ;
		\draw[->,thick,blue,] ([shift={(.3cm,-.6cm)}]left.south west) -- ([shift={(-.3cm,.5cm)}]right.north east) ;
	}
	\qquad
	M_3 = \left(\begin{array}{*2{c}}
	\tikzmark{left}{1} &{ 2} \\
	{3} &  \tikzmark{right}{4}  \\
	\tikzcontour{5} & \tikzcontour{6}\\
	\end{array}\right) 
	\]
	\Highlight[third]
	
	\[ M_1={3}\times{6}-{5}\times{4}  \hspace{.5cm} M_2=-({1}\times{6}-{5}\times{2}) \hspace{.5cm} M_3={1}\times{4}-{3}\times{2} \]
	\[ M_1=-2 \hspace{1.5cm} M_2=4 \hspace{1.5cm} M_3=-2 \]
	\[  \vec{u}(-2;4;-2)  \]
}%LR
\end{frame}
\begin{frame}[label=ex1]%EX3
\begin{exo}
	أحسب 
	$ 	\vec{u} \land \vec{v} $
		حيث 
$ \vec{u}\begin{pmatrix}1\\2\\-3\\	\end{pmatrix} $
و
  $ \vec{v}\ \begin{pmatrix}-2\\2\\5	\end{pmatrix} $
  \hyperlink{sol1}{\fbox{انقر هنا لتصحيح التمرين }}%SOL3
\end{exo}
\end{frame}




\section{ استقامية متجهتين}
\begin{frame}{خاصية}
	\begin{block}{خاصية}
			متجهتان مستقيمتان  يكافئ جدائهما المتجهي منعدم \\ بالرموز لدينا : \\
		\begin{align*}
		 \vec{u} \land \vec{v}=\vec{0} &\iff   \vec{u}=k \times \vec{v} \quad ou \quad \vec{v} =\vec{0}  \\
		\end{align*}
	\end{block}
\end{frame}
\begin{frame}[label=ex2]%ex4
	\begin{exo}
	
		 $ A\begin{pmatrix}0\\-1\\1\\	\end{pmatrix} $
		و
		$ B \begin{pmatrix}1\\1\\-2	\end{pmatrix} $
		و
		$ C \begin{pmatrix}-2\\1\\6	\end{pmatrix} $
		\\
		  	هل النقط A,B,C  مستقيمية ؟
		  	\\
		  \hyperlink{sol2}{\fbox{انقر هنا لتصحيح التمرين }}
	\end{exo}
\end{frame}

\section{تطبيقات}
\subsection{معادلة مستوى}
%\section{معادلة مستوى}
\begin{frame}{\hspace*{1.5cm}		معادلة مستوى معرف  بثلاث نقط }
	\begin{block}{\hspace*{1.5cm}	طريقة1 }
		\centering
	 $ \overrightarrow{AB} \land \overrightarrow{AC} $
	متجهة منظمية على المستوى (ABC)\\
	\LR{
	$ M(x,y,z) \in (ABC) \iff  \overrightarrow{AM} \times (\overrightarrow{AB} \land \overrightarrow{AC})=0 $
	}
	\end{block}
	\begin{block}{\hspace*{1.5cm}		طريقة2}	
	متجهة منظمية تحدد شكل معادلة المستوى ونقطة منه تحدد العدد التابث
	\end{block}
\end{frame}

\begin{frame}[label=ex3]{تمرين تطبيقي}%ex2
	%\begin{block}{\hspace*{1.5cm}		تمرين  : }
	\begin{exo}
	الفضاء منسوب ل م.م.م.م ونعتبر النقط 
	$ َA(1,0,0),B(0,1,0),C(0,0,1)  $
		\begin{dingautolist}{192}
			\item
			حدد احداثيات المتجهة 
			$ 	 \overrightarrow{AB} \land \overrightarrow{AC} $
			\item 
			استنتج ان النقط غير مستقيمية
			\item 
			تحقق ان  معادلة المستوى (ABC) هي :
			\LR{
			(ABC): x+y+z=1
		}
		\end{dingautolist}
	\hyperlink{sol3}{\fbox{انقر هنا لتصحيح التمرين }}
	%\hyperlink{sol1}{ \beamerbutton{تصحيح التمرين 	}}
	\end{exo}
	%\end{block}
	\hyperlink{sol3}{\beamerbutton{حل التمرين }}
\end{frame}%معادلة مستوى + تمرين 

\subsection{مسافة نقطةعن مستقيم}
%\section{مسافة نقطةعن مستقيم}
\begin{frame}%[allowframebreaks]
	%\subsection{Distance d'un point à une droite}
	\begin{block}{ مسافة نقطة عن مستقيم}
	مسافة  نقطة عن مستقيم هي المسافة الدنوية 
	التي تفصل هذه النقطة  عن أية نقطة من  المستقيم  
	
		  $d(M,\mathcal{D})=inf\left\lbrace MM', M' \in \mathcal{D}\right\rbrace $
	\end{block}
	\begin{block}{ خاصية}
		\begin{dingautolist}{192}
			\item 	(i) $d(M,\mathcal{D})=MH$ 
			مع
			 $H=proj_{\perp,\mathcal{D}}(M)$
			 المسقط العمودي للنقطة  
			 $ M $
			 على المستقيم
			 $  \mathcal{D} $
			\item 	(ii) $d(M,\mathcal{D})=\dfrac{\Vert \overrightarrow{AM} \wedge \overrightarrow{v}\Vert}{\Vert \overrightarrow{v} \Vert}$ 
			حيث
			 $\mathcal{D}(A,\overrightarrow{v})$
		\end{dingautolist}
	\end{block}
\end{frame}	

\begin{frame}

	\begin{block}{ملاحظة }
		ليكن
		 $\mathcal{P}$ 
		 المعرف بالنقط
		  $A,B,C$. 
		  لدينا
		   :\\
			$d(M,\mathcal{P})=\dfrac{\vert \overrightarrow{AM}.(\overrightarrow{AB} \wedge \overrightarrow{AC}) \vert}{\Vert \overrightarrow{AB} \wedge \overrightarrow{AC} \Vert}$
	\end{block}
	\pause
	\begin{proof}
	لان  المتجهة 
	 $\overrightarrow{AB}\wedge\overrightarrow{AC}$ 
	 منظمية على المستوى 
	 $ (ABC) $
	\end{proof}
\end{frame}
\begin{frame}[label=ex4]{تمرين تطبيقي}%ex2
	\begin{exo}
		الفضاء منسوب ل م.م.م.م ونعتبر النقط 
		$ َA(0,-1,1),B(1,1,-2),C(-2,1,6)  $
		\\
		احسب مسافة النقطة B عن المستقيم $ (AC) $
	
		\hyperlink{sol4}{\fbox{انقر هنا لتصحيح التمرين }}
		%\hyperlink{sol4}{ \beamerbutton{تصحيح التمرين 	}}
	\end{exo}
\hyperlink{sol4}{\beamerbutton{حل التمرين }}
\end{frame}




\subsection{‫مساحة ‬}
\begin{frame}{مساحة مثلث‬}
	\begin{block}{خاصية}
		$ S_{ABC} = \dfrac{1}{2} (\overrightarrow{AB}\land \overrightarrow{AC})$
		هي مساحة المثلث ABC
	\end{block}
	
	\begin{block}{ملحوظة }
		
		$ S =\overrightarrow{AB}\land \overrightarrow{AC}$
		هي مساحة متوازي الاضلاع   ABDC
	\end{block}
\end{frame}

%%%==============================================================
%\section{}\begin{frame}{}\end{frame}
\section{ اضافات}
\begin{frame}{صيغتا $ \cos$ و $ \sin $ }
	\begin{block}{خاصية}
		(i) $\cos(\overrightarrow{u},\overrightarrow{v})=\frac{\overrightarrow{u}.\overrightarrow{v}}{\Vert u \Vert \Vert v \Vert}$ 
		\hspace{2cm}(ii) $\vert \sin(\overrightarrow{u},\overrightarrow{v}) \vert=\frac{\Vert \overrightarrow{u} \wedge \overrightarrow{v} \Vert}{\Vert u \Vert \Vert v \Vert}$
	\end{block}
\end{frame}
\begin{frame}{موجهة تقاطع مستويين}
\begin{block}{خاصية}
	$\vec{n}\land \vec{n'}$
	موجهة لمستقيم تقاطع المستويين اللذين منظميتهما هما  
	$\vec{n}$ و $\vec{n'}$
\end{block}
مثال :
حدد موجهة تقاطع المستويين 
$ (D_1) $
و
$ (D\_2) $
بحيث 
\begin{align*}
(D_1):	2x-y+2z-3&=0 \\
(D_2):	3x+y-4z+1&=0
\end{align*}	
الحل :
\pause
\begin{block}{}
	$ (2;-1;2)\land (3;1;-4)=(2;14;5) $
\end{block}
\end{frame}
\section{ الحلول  } 
\begin{frame}{النهاية}
\begin{block}{نهاية الدرس}\huge
	انتهى الدرس و فيما يلي  حلول التمارين المدرجة فيه
\end{block}
	
\end{frame}
%\begin{tikzpicture}
%	\foreach \i in {1,2,...,4}{\include{solution\i}};
%\end{tikzpicture}
\begin{frame}[fragile,label=sol1]{حل التمرين} 

\begin{block}{ حل التمرين} 
	
		\begin{align*}
		\displaystyle \vec{u}\wedge \vec{v}=&\left |\begin{array}{cc} 2&2\\-3&5\end{array}\right |\vec{i}- \left |\begin{array}{cc}1&-2\\-3&5\end{array}\right |\vec{j}+ \left |\begin{array}{cc}  1&-2\\-3&5\end{array}\right |\vec{k} \\
		=&{\begin{pmatrix}16\\1\\6\end{pmatrix}}		
		\end{align*}
		  \hyperlink{ex1}{\fbox{انقر هنا للرجوع للتمرين }}

	
	\begin{tikzpicture}[remember picture, overlay]
	\node[shift={(-1cm,1cm)}]() at (current page.south east){%
		\hyperlink{ex1}{\beamerreturnbutton{\textarabic{الرجوع للتمرين}}}};  
	%\hyperlink{ex2}{\fbox{الرجوع للتمرين}}}};        
	\end{tikzpicture}
\end{block}
\end{frame}
\begin{frame}[fragile,label=sol2]{حل التمرين} 
% solution ex2
%\hypertarget<2->{ex2}{\beamerbutton{هناتصحيح  التمرين 2 }}
\begin{block}{ حل التمرين} 
	 $ \V{AB}\begin{pmatrix}1\\2\\-3\\	\end{pmatrix} $
	و
	$ \V{AC} \begin{pmatrix}-2\\2\\5	\end{pmatrix} $
	
	
		\begin{align*}
	\displaystyle \V{AB}\wedge \V{AC}={\begin{pmatrix}16\\1\\6\end{pmatrix}}
	\neq  \vec{0}		
		\end{align*}
		اذن المتحهتان 
		 $ \V{AB} $
		و
		$ \V{AC} $
		غير مستقيميتان
		ومنه النقط A,B,C غير مستقيمية
		  \hyperlink{ex2}{\fbox{انقر هنا للرجوع للتمرين }}

	
	\begin{tikzpicture}[remember picture, overlay]
	\node[shift={(-1cm,1cm)}]() at (current page.south east){%
		\hyperlink{ex2}{\beamerreturnbutton{\textarabic{الرجوع للتمرين}}}};  
	%\hyperlink{ex2}{\fbox{الرجوع للتمرين}}}};        
	\end{tikzpicture}
\end{block}
\end{frame}
\begin{frame}[plain,label=sol3]{ حل التمرين} %[allowframebreaks]
\begin{block}{حل التمرين} 
	\begin{dingautolist}{192}
		\item 
		\begin{align*}
		\overrightarrow{AB} \land \overrightarrow{AC} =&\begin{pmatrix}-1 \\1 \\ 0
		\end{pmatrix} \land \begin{pmatrix}-1 \\0 \\ 1
		\end{pmatrix}
		\\
		=& \left |\begin{array}{cc} 1&0\\0&1\end{array}\right |\vec{i}- \left |\begin{array}{cc} -1&-1\\0&1\end{array}\right |\vec{j}+ \left |\begin{array}{cc} -1&-1\\1&0\end{array}\right |\vec{k} \\
		=&+\vec{i}+\vec{j}+\vec{k} \\
		=&\begin{pmatrix}1 \\1\\ 1 \end{pmatrix}
		\end{align*}
		\item 
		$\quad \iff \overrightarrow{AB} \land \overrightarrow{AC} \neq \vec{0}  $
		النقط A,B,C غيرمستقيمية
	\end{dingautolist}
\hyperlink{ex3}{\fbox{\textarabic{الرجوع للتمرين}}}
\hyperlink{}{\fbox{\textarabic{تتمة حل التمرين}}}
\end{block}
\end{frame}


\begin{frame}[fragile,label=sol3bis]
\begin{block} {حل التمرين}
 	\begin{align*}
	M(x,y,z) \in (ABC) & ~\iff  \overrightarrow{AM} \times (\overrightarrow{AB} \land \overrightarrow{AC})=0  \\
	&\iff \begin{pmatrix}x-1 \\y\\ z\end{pmatrix} \times \begin{pmatrix}1 \\1\\ 1
	\end{pmatrix}=0 \\
	& \iff x-1+y+z=0 \\
	& (ABC):x+y+z=1
	\end{align*} 
	\hyperlink{sol3bisautre}{\fbox{\textarabic{الطريقة الثانية}}}
	\hyperlink{ex3}{\fbox{\textarabic{الرجوع للتمرين}}}
\end{block}
%	\begin{tikzpicture}[remember picture,overlay]
%	\node[shift={(-1cm,1cm)}]() at (current page.south east){%
%		\hyperlink{sol3bis}{\beamerreturnbutton{\textarabic{الطريقة التانية}}}};  
%%\hyperlink{ex3}{\fbox{الرجوع للتمرين}}}};        
%	\end{tikzpicture}
\end{frame}
	\begin{frame}[label=sol3bisautre]
	\begin{block} {الطريقة التانية}
		$ \overrightarrow{AB} \land \overrightarrow{AC}=\begin{pmatrix}1 \\1\\ 1
		\end{pmatrix} $
		منظمية على المستوى $ (ABC) $ 
		
		اذن معادلة المستوى تكتب على شكل 
		$ (ABC):x+y+z+d=0 $
		
		وبماان 
		 $ A \in (ABC) $
		وبتعويض احداثيات A في هذه المعادلة نحصل على 
		
		$ 1+0+0+d=0 $
		أي
		$ d=-1 $
		
		ومنه معادلة المستوى هي :

			$ (ABC):x+y+z-1=0 $
		 
		\hyperlink{ex3}{\fbox{\textarabic{الرجوع للتمرين}}}
	\end{block}
\end{frame}
\begin{frame}[fragile,label=sol4]{حل التمرين} 
	\begin{block}{} 
			\begin{dingautolist}{192}
				\item
				لدينا 
				
				$ 	 \overrightarrow{AB} \land \overrightarrow{AC}=\left |\begin{array}{cc} 2&2\\-3&5\end{array}\right |\vec{i}- \left |\begin{array}{cc}1&-2\\-3&5\end{array}\right |\vec{j}+ \left |\begin{array}{cc}  1&-2\\-3&5\end{array}\right |\vec{k}
			= \begin{pmatrix}16\\1\\6\end{pmatrix} $
					\item 
					$d(B,(AC))=\dfrac{\Vert \overrightarrow{AB} \wedge \overrightarrow{AC}\Vert}{\Vert \overrightarrow{AC} \Vert}=\dfrac{\sqrt{16^2+1^2+6^2}}{\sqrt{(-2)^2+2^2+5^2}}=\sqrt{\dfrac{293}{33}}$ 	
			\end{dingautolist} 
			\hyperlink{ex4}{\fbox{انقر هنا للرجوع للتمرين}}
	\end{block}
\end{frame}
	\end{document}%%%==============================================================%%%==============================================================%%%==============================================================%%%==============================================================%%%==============================================================
\begin{frame}[label=sol]{	تصحيح التمرين }
\begin{exo}
	
\end{exo}
\end{frame}

=====================
\section{ }\begin{frame}{}

\end{frame}
\only<1>{% solution ex1
	\begin{block}{1 حل التمرين} 
		
	\end{block}
}
=====================
\section{}\begin{frame}{}

\end{frame}

https://fr.wikipedia.org/wiki/Produit_vectoriel#/media/Fichier:Technique_de_calcul_du_produit_vectoriel.gif

%%%%%%%%%%%%%%%%%  codes reutilisables  %%%%%%%%%%%%%%%%%
\begin{small}
\begin{columns}
	\begin{column}{0.3\linewidth}
		\begin{itemize}
			\item[$\textcolor{red!90!black}{\bigstar}$]
			AA
		\end{itemize}
	\end{column}
	\begin{column}{0.7\linewidth}
	\includegraphics[width=\linewidth]{example-image}
\end{column}
\end{columns}
\end{small} 
%%%%%%%%%%%%%%%%%%%
\subsection{Distance entre deux points}
Soient $A(x_{A},y_{A},z_{A})$, $B(x_{B},y_{B},z_{B})$. Alors $d(A,B)=\Vert \overrightarrow{AB} \Vert=\sqrt{\overrightarrow{AB}.\overrightarrow{AB}}=\sqrt{(x_{B}-x_{A})^{2}+(y_{B}-y_{A})^{2}+(z_{B}-z_{A})^{2}}$

%%%%%%%%%%%%%%%%%%%%%%%%%%%%%%%%%%%%%%%%%%%%%%%%%%%%%%%%%%%%%%%%%%%
\setbeamertemplate{background canvas}{\includegraphics[width=\paperwidth,height=\paperheight]{example-image}}

%%%%%%%%%%%%%%%%%%%%%%%%%%%%%%%%%%%%%%%%%%%%%%%%%%%%%%%%%%%%%
\begin{multicols}{2}
	\begin{itemize}[<+->]
		\item First.
		\item Second.
		\item Third.
		\item Fourth.\hypertarget<4>{label}{\beamerbutton{I'm on the fourth slide}}
		\item Fifth.
	\end{itemize}
	\columnbreak
	autre colonne
\end{multicols}


%%%%%%%%%%%%%%%%%%%%%%%%%%%%%%%%%%%%%%%%%%%%%%%%
\begin{frame}{title}
\begin{multicols}{2}%package multicol
	\begin{minipage}{.5\textwidth}
		\centering
		gauche	 in minipage
	\end{minipage}
	\vfill
	\begin{minipage}{.5\textwidth}
		\centering
		droite in mini page
	\end{minipage}
\end{multicols}
content
\end{frame}
%%%%%%%%%%%%%%%%%%%%%%%%%%%%%%%%%%%%%%%%%%%%%%%%
%\begin{small}
\begin{columns}
	\begin{column}{0.3\linewidth}
		\begin{itemize}
			\item[$\textcolor{red!90!black}{\bigstar}$]
			AA
			\item[$\textcolor{red!90!black}{\bigstar}$]
			AA
			\item[$\textcolor{red!90!black}{\bigstar}$]
			AA
		\end{itemize}
	\end{column}
	\begin{column}{0.7\linewidth}
		\includegraphics[width=\linewidth]{example-image}
	\end{column}
\end{columns}
%\end{small} 

%%%%%%%%%%%%%%%%%%%%%%%%  exo et sa  solution  %%%%%%%%%%%%
%%%%%%%%%%%%%%%  model cours  exo sol     %%%%%%%%%%%%%%%%%%%%%%

%\section{ title}
%\subsection{title}
\begin{frame}[option]{titlecours}
content
\begin{block}{titre}
	content
\end{block}
\end{frame}
\begin{frame}[label=ex4]{تمرين تطبيقي}
content
\begin{exo}
	content
	\begin{dingautolist}{192}
		\item	
	\end{dingautolist} 
\begin{block}{titre}
	content
\end{block}

\hyperlink{sol4}{\fbox{انقر هنا لتصحيح التمرين }}
%\hyperlink{sol4}{ \beamerbutton{تصحيح التمرين 	}}
\end{exo}

\hyperlink{sol4}{\beamerbutton{حل التمرين }}
\end{frame}


\begin{frame}[fragile,label=sol4]{حل التمرين} 
% solution ex4
%\hypertarget<2->{ex2}{\beamerbutton{هناتصحيح  التمرين  }}
\begin{block}{ حل التمرين} 

\begin{dingautolist}{192}
\item

\end{dingautolist} 
\hyperlink{ex4}{\fbox{انقر هنا للرجوع للتمرين }}
\end{block}

\end{frame}

%%%%%%%%%%%%%%% fin model cours  exo sol     %%%%%%%%%%%%%%%%%%%%%%


