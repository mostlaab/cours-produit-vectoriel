\begin{frame}{توجيه الفضاء-  توجيه المستوى }
%\title{	الجداء المتجهي}
\begin{minipage} {.75\textwidth}
	

	\begin{small}
		\begin{block}{}
			
		

	ليكن 
	 $\mathcal{R}=(O,\vec{i},\vec{j},\vec{k})$
	 معلما للفضاء أي النقط 
	O,I,J,K 
	 بحيث : \\
	$\vec{i}=\V{OI},\vec{j}=\V{OJ}, \vec{k}=\V{OK}$ 
	نقط غير مستوائية 
	
	رجل امبير للمعلم 
	 $\mathcal{R}$ 
	 هو رجل خيالي راسه في K قدماه في O  وينظر الى I 
	 
	إذاكان J   على يسار هدا الرجل فإن المعلم  يكون مباشراً او موجباً
	
	إداكان المعلم  مباشر فان الاساس
	 	 $(\vec{i},\vec{j},\vec{k})$
	 	 مباشر
	 	 
	الفضاء 
	$(\mathcal{E})$
	 موجه إذاكان منسوباً إلى معلم مباشر
	 
	ملحوظة :
	
	‫يمكن إنشاء  معلم ‬مباشر 
		 $\mathcal{R}=(O,\vec{i},\vec{j},\vec{k})$ 
		  بحيث‬‬ 
$ 	\vec{i},\vec{j} $
	موجهتين  لمستوى 
%	$ (\mathcal{P}) $

	وتوجيه المستوى بتوجيه منظمية عليه 
\end{block}


%\begin{tikzpicture}[remember picture,overlay]
%%	\node at ([xshift=-5cm]current page){ ooo};
%	\begin{scope}[xshift=1cm,yshift=3cm]
%\draw[->,latex] (0,0)node[left]{\tiny O}--node[pos=0.5,left]{$\vec{ k}$} (0,1) node[above]{\tiny K} ;
%\draw[->,latex]		(0,0)-- node[pos=0.5,below]{$\vec{ j}$}(1,0)node[right]{\tiny J} ;
%\draw[->,latex]		(0,0)-- node[pos=0.5,below]{$\vec{ i}$}(-.5,-.5)node[left]{\tiny I};
%
%	\end{scope}
%\end{tikzpicture}
\end{small}
\end{minipage}
\end{frame}
\begin{frame}

\begin{minipage}{.75\textwidth}

	\begin{block}{تعريف  الجداء المتجهي :}
	
	ليكن 
	$ 	\vec{u}=\V{OA},\vec{v}=\V{OB}$
	 متجهتين من الفضاء المتجهي
	  
	الجداء المتجهي للمتجهتين 
		$ 	\vec{u}$ و $\vec{v}$
	بهذا الترتيب هو المتجهة 
	 
	$	\vec{w}= \vec{u} \land \vec{v}$
	
	 المعرفة بمايلي:
	 
	 		$	\vec{w}=\V{OC}$
	 		  
	 			  \begin{enumerate}\RTListe
	 			\item 
	اذاكان 	
	$ 	\vec{u}$ و $\vec{v}$ 
	مستقيميتان فان
		 		$	\vec{w}=\vec{0}$
		 		\item 
	اذاكان 
		$ 	\vec{u}$ و $\vec{v}$
	 غير مستقيميتين فان المتجهة
$\vec{w} $
	  تحقق
	
	  \begin{enumerate}\RTListe
	  	\item 
	$ (OA) 	 \perp	(OC)  $ و	$ (OC) \perp  (OB) $
	\item 
	$  (	\vec{u},\vec{v},\vec{w}) $
	اساس مباشر 
	\item 
$ 	OC= OA \times OB \times | \sin ( \V{OA} ,\V{OB} )| $
	  \end{enumerate}
	  \end{enumerate}

	\end{block}
	
\end{minipage}	
	
%	$ \vec{ i}   \land \vec{j }  = \vec{k } $ \\
%$	\vec{j }  \land  \vec{k } = \vec{i } $\\
%$	\vec{k }  \land  \vec{i }  = \vec{ j} $

%	احسب دائرية جداءات متجهات الاساس
%	التاويل الهندسي لمنظم الجداء المتجهي هو مساحة متوازي اضلاع و منه مساحة مثلث 
\end{frame}
\begin{frame}{خاصيات الجداء المتجهي }
\begin{small}
\begin{minipage}{.75\textwidth}
	
\begin{block}{خاصية}
	\centering
$ \forall \alpha ,\beta \in \rr ,\forall \vec{ u},\vec{ v},\vec{ w} \in \mathcal{V}_3 $ \vspace{.5cm}
\begin{enumerate}[t]\RTListe
	\setlength \itemsep{.5cm}
\item 
$ \vec{u }  \land\vec{v }  = -\vec{v } \land\vec{u }  $
\item 
$ \alpha\beta (\vec{ u} \land\vec{v } )=(\alpha \vec{ u} ) \land(\beta\vec{v })     $  
\item 
$ \vec{u }  \land(\vec{ v}  + \vec{ w}  )=\vec{ u}  \land\vec{v }  + \vec{u}  \land \vec{ w} $
\end{enumerate}
\end{block}
\end{minipage}


ملحوظة : 

اذا كان  
$(\vec{i},\vec{j},\vec{k})$
اساس متعامد ممنظم ومباشر  فإن :

\begin{tabular}{l|l}
%	\hline 
	$ \vec{ i}   \land \vec{i } =\vec{ 0} $	& 	$ \vec{ i}   \land \vec{j }  = \vec{k } $ \\ 
	%		\hline 
	$ \vec{ j}   \land \vec{j } =\vec{ 0}  $	& $	\vec{j }  \land  \vec{k } = \vec{i } $ \\ 
	%		\hline 
	$ \vec{ k}   \land \vec{k } =\vec{ 0}  $	&$	\vec{k }  \land  \vec{i }  = \vec{ j} $  \\ 
%	\hline 
\end{tabular} 

\end{small}
\end{frame}

\begin{frame}[allowframebreaks]{تطبيق :
	الصيغة التحليلية للجداء المتجهي في م.م.م.م }

\begin{block}{}
	

	تحديد احداثيات :
$ 	  \vec{u}\wedge \vec{v}= \begin{pmatrix}u_{1}\\u_{2}\\u_{3}\
\end{pmatrix}\land   \begin{pmatrix}v_{1}\\v_{2}\\v_{3}\\	\end{pmatrix} $
في معلم متعامد ممنظم مباشر  $\mathcal{R}=(O,\vec{i},\vec{j},\vec{k})$

\end{block}
\begin{small}


		\begin{align*}
	\displaystyle \vec{u}\wedge \vec{v}=& (u_{1}\vec{i}+u_{2}\vec{j}+u_{3}\vec{k})
\land  (v_{1}\vec{i}+v_{2}\vec{j}+v_{3}\vec{k})\\
	=&\stackrel{\vec{ 0}}{\overbrace{u_{1}\vec{i}\land v_{1}\vec{i}}} +\stackrel{u_{1}v_{2}\vec{ k}}{\overbrace{u_{1}\vec{i}\land v_{2}\vec{j}}}+\stackrel{-u_{1}v_{3}\vec{j}}{\overbrace{u_{1}\vec{i}\land v_{3}\vec{k}}}+\\
	 &+\stackrel{-u_{2}v_{1}\vec{ k}}{\overbrace{u_{2}\vec{j}\land v_{1}\vec{i}}} +\stackrel{\vec{ 0}}{\overbrace{u_{2}\vec{j}\land v_{2}\vec{j}}}+\stackrel{u_{2}v_{3}\vec{ i}}{\overbrace{u_{2}\vec{j}\land v_{3}\vec{k}}}+\\	
	 &+\stackrel{u_{3}v_{1}\vec{ j}}{\overbrace{u_{3}\vec{k}\land v_{1}\vec{i}}} +\stackrel{-u_{3}v_{2}\vec{ i}}{\overbrace{u_{3}\vec{k}\land v_{2}\vec{j}}}+\stackrel{\vec{ 0}}{\overbrace{u_{3}\vec{k}\land v_{3}\vec{k}}}\\
	=&{(u_{2}v_{3}-u_{3}v_{2})\vec{i}+(\u_{3}v_{1}-u_{1}v_{3})\vec{j}+(u_{1}v_{2}-u_{2}v_{1})\vec{k}}	\\
%	\end{align*}	
%	ومنه نستنتج :
%	
%		\begin{align*}
	=&\left |\begin{array}{cc} u_{2}&v_{2}\\u_{3}&v_{3}\end{array}\right |\vec{i}- \left |\begin{array}{cc}u_{1}&v_{1}\\u_{3}&v_{3}\end{array}\right |\vec{j}+ \left |\begin{array}{cc}  u_{1}&v_{1} \\u_{2}&v_{2}\end{array}\right |\vec{k} \\
	=&{\begin{pmatrix}u_{2}v_{3}-u_{3}v_{2}\\u_{3}v_{1}-u_{1}v_{3}\\u_{1}v_{2}-u_{2}v_{1}\end{pmatrix}}		
	\end{align*}
\end{small}
\end{frame}