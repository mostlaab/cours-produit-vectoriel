%\section{ تعريف }
%% تعريف وامثلة مع الشكل 
\begin{frame}%[plain]%{احدثيات الجداء المتجهي}
\begin{block}{احداثيات الجداء المتجهي في .م.م.م.م}
	\begin{align*}
\displaystyle \vec{u}\wedge \vec{v}=& \begin{pmatrix}u_{1}\\u_{2}\\u_{3}\\
\end{pmatrix}\land   \begin{pmatrix}v_{1}\\v_{2}\\v_{3}\\	\end{pmatrix}\\
=&\left |\begin{array}{cc} u_{2}&v_{2}\\u_{3}&v_{3}\end{array}\right |\vec{i}- \left |\begin{array}{cc}u_{1}&v_{1}\\u_{3}&v_{3}\end{array}\right |\vec{j}+ \left |\begin{array}{cc}  u_{1}&v_{1} \\u_{2}&v_{2}\end{array}\right |\vec{k} \\
=&{\begin{pmatrix}u_{2}v_{3}-u_{3}v_{2}\\u_{3}v_{1}-u_{1}v_{3}\\u_{1}v_{2}-u_{2}v_{1}\end{pmatrix}}		
\end{align*}
%%% le lien suivant ne marche pas dans https://lewebpedagogique.com/most
%	\href{https://upload.wikimedia.org/wikipedia/commons/thumb/b/b2/Technique_de_calcul_du_produit_vectoriel.gif/220px-Technique_de_calcul_du_produit_vectoriel.gif}{ \fbox{انقرهنا }}
\end{block}
\end{frame}
\begin{frame}{مثال :}
\begin{block}{}
	 
	\begin{center}
	 $\vec{u} =\begin{pmatrix}{1}\\{3}\\{5}\\
	\end{pmatrix}\land   \begin{pmatrix}{2}\\{4}\\{6}\\	\end{pmatrix}= \begin{pmatrix}{M_1}\\{M_2}\\{M_3}\\ \end{pmatrix} $
\end{center}
\end{block}
حيث
\LR{
	\[
	M_1 = \left(\begin{array}{*2{c}}
	\tikzcontour{1} &\tikzcontour {2 }\\
	\tikzmark{left} {3} & 4  \\
	5 &  \tikzmark{right}{6}\\
	\end{array}\right) \Highlight[first]
	\quad
	M_2 = \left(\begin{array}{*2{c}}
	\tikzmark{left}   {1} & 2 \\
	\tikzcontour{3} & \tikzcontour{4} \\
	5 &  \tikzmark{right}{6}\\
	\end{array}\right)  \Highlight[second]
	\tikz[overlay,remember picture] {
		\draw[->,thick,red,] ([shift={(-.1cm,.1cm)}]left.south east) -- ([shift={(.1cm,-.2cm)}]right.north west) ;
		\draw[->,thick,blue,] ([shift={(.3cm,-.6cm)}]left.south west) -- ([shift={(-.3cm,.5cm)}]right.north east) ;
	}
	\qquad
	M_3 = \left(\begin{array}{*2{c}}
	\tikzmark{left}{1} &{ 2} \\
	{3} &  \tikzmark{right}{4}  \\
	\tikzcontour{5} & \tikzcontour{6}\\
	\end{array}\right) 
	\]
	\Highlight[third]
	
	\[ M_1={3}\times{6}-{5}\times{4}  \hspace{.5cm} M_2=-({1}\times{6}-{5}\times{2}) \hspace{.5cm} M_3={1}\times{4}-{3}\times{2} \]
	\[ M_1=-2 \hspace{1.5cm} M_2=4 \hspace{1.5cm} M_3=-2 \]
	\[  \vec{u}(-2;4;-2)  \]
}%LR
\end{frame}
\begin{frame}[label=ex1]%EX3
\begin{exo}
	أحسب 
	$ 	\vec{u} \land \vec{v} $
		حيث 
$ \vec{u}\begin{pmatrix}1\\2\\-3\\	\end{pmatrix} $
و
  $ \vec{v}\ \begin{pmatrix}-2\\2\\5	\end{pmatrix} $
  \hyperlink{sol1}{\fbox{انقر هنا لتصحيح التمرين }}%SOL3
\end{exo}
\end{frame}


