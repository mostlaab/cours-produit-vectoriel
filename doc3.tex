%\section{معادلة مستوى}
\begin{frame}{\hspace*{1.5cm}		معادلة مستوى معرف  بثلاث نقط }
	\begin{block}{\hspace*{1.5cm}	طريقة1 }
		\centering
	 $ \overrightarrow{AB} \land \overrightarrow{AC} $
	متجهة منظمية على المستوى (ABC)\\
	\LR{
	$ M(x,y,z) \in (ABC) \iff  \overrightarrow{AM} \times (\overrightarrow{AB} \land \overrightarrow{AC})=0 $
	}
	\end{block}
	\begin{block}{\hspace*{1.5cm}		طريقة2}	
	متجهة منظمية تحدد شكل معادلة المستوى ونقطة منه تحدد العدد التابث
	\end{block}
\end{frame}

\begin{frame}[label=ex3]{تمرين تطبيقي}%ex2
	%\begin{block}{\hspace*{1.5cm}		تمرين  : }
	\begin{exo}
	الفضاء منسوب ل م.م.م.م ونعتبر النقط 
	$ َA(1,0,0),B(0,1,0),C(0,0,1)  $
		\begin{dingautolist}{192}
			\item
			حدد احداثيات المتجهة 
			$ 	 \overrightarrow{AB} \land \overrightarrow{AC} $
			\item 
			استنتج ان النقط غير مستقيمية
			\item 
			تحقق ان  معادلة المستوى (ABC) هي :
			\LR{
			(ABC): x+y+z=1
		}
		\end{dingautolist}
	\hyperlink{sol3}{\fbox{انقر هنا لتصحيح التمرين }}
	%\hyperlink{sol1}{ \beamerbutton{تصحيح التمرين 	}}
	\end{exo}
	%\end{block}
	\hyperlink{sol3}{\beamerbutton{حل التمرين }}
\end{frame}