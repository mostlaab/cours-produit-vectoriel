%\section{مسافة نقطةعن مستقيم}
\begin{frame}%[allowframebreaks]
	%\subsection{Distance d'un point à une droite}
	\begin{block}{ مسافة نقطة عن مستقيم}
	مسافة  نقطة عن مستقيم هي المسافة الدنوية 
	التي تفصل هذه النقطة  عن أية نقطة من  المستقيم  
	
		  $d(M,\mathcal{D})=inf\left\lbrace MM', M' \in \mathcal{D}\right\rbrace $
	\end{block}
	\begin{block}{ خاصية}
		\begin{dingautolist}{192}
			\item 	(i) $d(M,\mathcal{D})=MH$ 
			مع
			 $H=proj_{\perp,\mathcal{D}}(M)$
			 المسقط العمودي للنقطة  
			 $ M $
			 على المستقيم
			 $  \mathcal{D} $
			\item 	(ii) $d(M,\mathcal{D})=\dfrac{\Vert \overrightarrow{AM} \wedge \overrightarrow{v}\Vert}{\Vert \overrightarrow{v} \Vert}$ 
			حيث
			 $\mathcal{D}(A,\overrightarrow{v})$
		\end{dingautolist}
	\end{block}
\end{frame}	

\begin{frame}

	\begin{block}{ملاحظة }
		ليكن
		 $\mathcal{P}$ 
		 المعرف بالنقط
		  $A,B,C$. 
		  لدينا
		   :\\
			$d(M,\mathcal{P})=\dfrac{\vert \overrightarrow{AM}.(\overrightarrow{AB} \wedge \overrightarrow{AC}) \vert}{\Vert \overrightarrow{AB} \wedge \overrightarrow{AC} \Vert}$
	\end{block}
	\pause
	\begin{proof}
	لان  المتجهة 
	 $\overrightarrow{AB}\wedge\overrightarrow{AC}$ 
	 منظمية على المستوى 
	 $ (ABC) $
	\end{proof}
\end{frame}
\begin{frame}[label=ex4]{تمرين تطبيقي}%ex2
	\begin{exo}
		الفضاء منسوب ل م.م.م.م ونعتبر النقط 
		$ َA(0,-1,1),B(1,1,-2),C(-2,1,6)  $
		\\
		احسب مسافة النقطة B عن المستقيم $ (AC) $
	
		\hyperlink{sol4}{\fbox{انقر هنا لتصحيح التمرين }}
		%\hyperlink{sol4}{ \beamerbutton{تصحيح التمرين 	}}
	\end{exo}
\hyperlink{sol4}{\beamerbutton{حل التمرين }}
\end{frame}


