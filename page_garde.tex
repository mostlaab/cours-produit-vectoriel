\begin{frame}[plain]\transsplitverticalout
\begin{tcolorbox}[colback=white,drop large lifted shadow,top=.5cm,bottom=.5cm]
	{\begin{center}
			{\large{بسم الله الرحمان الرحيم}} %\emph\ttfamily
			\maketitle
	\end{center}}
\end{tcolorbox}
\end{frame}

\begin{frame}[plain]{\large \textbf{\huge	تتمة درس الهندسة الفضائية	:}}
\transglitter[duration=0.5]

\begin{tcolorbox}[colback=white,drop large lifted shadow,top=.5cm,bottom=.5cm]
\begin{center}
	\begin{LARGE}
		{\color{blue}\huge
			{  الجداء المتجهي - تطبيقات
		}}
	\end{LARGE}
\end{center}
\end{tcolorbox}



\begin{itemize}%[<+->]
	\RTListe %pour aligner à droite les items
\item[$\textcolor{red}{\surd}$] 
تعريف الجداء المتجهي 
\item 
انعدام الجداء المتجهي  واستقامية متجهتين
\item [$\textcolor{red}{\surd}$] 
خاصيات
\item [$\textcolor{red}{\surd}$] 
المعلم المباشر

\item [$\textcolor{red}{\surd}$] 
احداثيات الجداء المتجهي لمتحهتين
\item [$\textcolor{red}{\surd}$] 
تطبيقات  الجداء المتجهي :

\begin{itemize}%[<+->]
	\RTListe
	\item 
	معادلة مستوى معرف بثلاث نقط
	\item 
	مسافة نقطة عن مستقيم
	\item 
	مساحة مثلث
		\item 
	موجهة تقاطع مستويين
\end{itemize}
\end{itemize}
\end{frame}
