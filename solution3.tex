\begin{frame}[plain,label=sol3]{ حل التمرين} %[allowframebreaks]
\begin{block}{حل التمرين} 
	\begin{dingautolist}{192}
		\item 
		\begin{align*}
		\overrightarrow{AB} \land \overrightarrow{AC} =&\begin{pmatrix}-1 \\1 \\ 0
		\end{pmatrix} \land \begin{pmatrix}-1 \\0 \\ 1
		\end{pmatrix}
		\\
		=& \left |\begin{array}{cc} 1&0\\0&1\end{array}\right |\vec{i}- \left |\begin{array}{cc} -1&-1\\0&1\end{array}\right |\vec{j}+ \left |\begin{array}{cc} -1&-1\\1&0\end{array}\right |\vec{k} \\
		=&+\vec{i}+\vec{j}+\vec{k} \\
		=&\begin{pmatrix}1 \\1\\ 1 \end{pmatrix}
		\end{align*}
		\item 
		$\quad \iff \overrightarrow{AB} \land \overrightarrow{AC} \neq \vec{0}  $
		النقط A,B,C غيرمستقيمية
	\end{dingautolist}
\hyperlink{ex3}{\fbox{\textarabic{الرجوع للتمرين}}}
\hyperlink{}{\fbox{\textarabic{تتمة حل التمرين}}}
\end{block}
\end{frame}


\begin{frame}[fragile,label=sol3bis]
\begin{block} {حل التمرين}
 	\begin{align*}
	M(x,y,z) \in (ABC) & ~\iff  \overrightarrow{AM} \times (\overrightarrow{AB} \land \overrightarrow{AC})=0  \\
	&\iff \begin{pmatrix}x-1 \\y\\ z\end{pmatrix} \times \begin{pmatrix}1 \\1\\ 1
	\end{pmatrix}=0 \\
	& \iff x-1+y+z=0 \\
	& (ABC):x+y+z=1
	\end{align*} 
	\hyperlink{sol3bisautre}{\fbox{\textarabic{الطريقة الثانية}}}
	\hyperlink{ex3}{\fbox{\textarabic{الرجوع للتمرين}}}
\end{block}
%	\begin{tikzpicture}[remember picture,overlay]
%	\node[shift={(-1cm,1cm)}]() at (current page.south east){%
%		\hyperlink{sol3bis}{\beamerreturnbutton{\textarabic{الطريقة التانية}}}};  
%%\hyperlink{ex3}{\fbox{الرجوع للتمرين}}}};        
%	\end{tikzpicture}
\end{frame}
	\begin{frame}[label=sol3bisautre]
	\begin{block} {الطريقة التانية}
		$ \overrightarrow{AB} \land \overrightarrow{AC}=\begin{pmatrix}1 \\1\\ 1
		\end{pmatrix} $
		منظمية على المستوى $ (ABC) $ 
		
		اذن معادلة المستوى تكتب على شكل 
		$ (ABC):x+y+z+d=0 $
		
		وبماان 
		 $ A \in (ABC) $
		وبتعويض احداثيات A في هذه المعادلة نحصل على 
		
		$ 1+0+0+d=0 $
		أي
		$ d=-1 $
		
		ومنه معادلة المستوى هي :

			$ (ABC):x+y+z-1=0 $
		 
		\hyperlink{ex3}{\fbox{\textarabic{الرجوع للتمرين}}}
	\end{block}
\end{frame}